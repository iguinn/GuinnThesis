\makeatletter
\def\input@path{{../}}
\makeatother

\documentclass[/main.tex]{subfiles}

\begin{document}

\textpages
\chapter{Simulating the \MJD}

\section{Introduction}


\section{Simulation Software}
\subsection{\Mage}
\Mage\ (Majorana/Gerda) \cite{mage2011} is a Monte Carlo software package developed jointly by the \MJ\ and \Gerda\ collaborations for the purpose of simulating low-background experiments involving HPGe detectors.
\Mage is written primarily in \cpp\ and is based on the \geant\ physics simulation framework\cite{geant2003}.
A \geant\ simulation requires the following inputs:
\begin{itemize}
\item{\textbf{Experiment Geometry:}}
  A discription of the physical dimensions, location, and materials must be provided.
  These should be included for both the detectors and the experimental structure surrounding the detectors.
\item{\textbf{Event Generator:}}
  A generator creates the initial conditions for an event, including the initial particles generated, and the intitial positions and momenta of each initial particle.
  The initial positions will typically be sampled from a particular subset of the full experimental geometry, such as the volume defined by a particular component.
  The initial momenta will be sampled from the allowed phase space of the process, conserving energy and momentum and sampling the angular correlation distribution.
  Many processes will generate multiple events; for example, a \Th{228} decay will generate a set of particles for each decay in the chain, including $\gamma$s generated by nuclear deexcitations.
\item{\textbf{Physics Lists:}}
  The physics lists describe the physical processes to be simulated as the generated particles propagate through the experimental geometry.
  Examples of such processes include Compton scattering of $\gamma$-rays in matter and energy deposition of electrons as they propagate through matter.
  A physics list will describe the probability of a process happening in a given material, any changes to the particle that generated the process, and any new particles produced by the process.
\end{itemize}
\Mage\ contains geometries describing various detector configurations for the \MJD.
\Mage\ also includes event generators that are used to describe \bb from inside the detectors, backgrounds generated by various experimental components from various isotopes, and the line sources used in detector calibration.
Finally, \Mage\ includes the relevant physics lists to the nuclear processes observed by the \MJD.
\Mage\ enables a user to select a geometry and event generator by writing a simple macro and running the \texttt{MaGe} executable on that macro.
\\
A Monte Carlo run by \geant\ will generate a large number of event primaries.
A Monte Carlo event primary begins with the set of particles created by the input event generator.
Each particle will then be given a particle track, describing the path it takes through the experiment geometry.
If the particle undergoes an interaction with the experiment as described in a physics list, the particle track will end and a Monte Carlo step will occur.
A Monte Carlo step describes the particle before and after an interaction, any additional particles generated in the interaction, and the amount of energy imparted into the matter at the site of the interaction.
For each Monte Carlo event primary, \Mage\ will record an event with the details of each Monte Carlo step that occurs inside of a detector, including the position of the step, the incoming particle, the outgoing particles, the physics process that caused the step, and the amount of energy deposited.
If no interactions occur inside of a detector, no event will not be recorded, but recorded events are enumerated according to the event primaries to ensure that the detection efficiency can be accurately counted.
\\
The simulation events are stored in a \TTree\ containing the following branches:
\begin{itemize}
\item{\texttt{fMCRun}:}
  Contains meta-information about the simulation run, including the run number, number of events, and settings for the run.
\item{\texttt{eventHeader}:}
  Contains meta-information about the event such as the event ID.
\item{\texttt{eventSteps}:}
  Contains data from each event step that deposits energy in a HPGe detector volume, including the location, process and energy deposition.
\item{\texttt{eventPrimaries}:}
  Contains data from the first step in an event, which generated the event.
\end{itemize}

\subsection{Simulation Post-Processing}
Once a \Mage\ simulation is run, the data generated must be post-processed to look like \MJD\ data.
Post-processing is performed by the GAT executable \texttt{process\_MJD\_as\_built\_mage\_results}.
The post-processor requires the input of individual detector characteristics such as energy resolution and dead layer parameters, which are provided via a JSON file.
The relevant steps of the posts-processor will be described in the next few paragraphs.
\\
First, steps within 0.1~mm and 5~ns of each other are grouped into clusters.
A typical cluster will contain the initial physics process that generated the cluster, such as a Compton scatter or $\beta$ decay which generate a high energy electron in the detector, and many electron scatters as the electron comes to rest inside of the detector, generating a cloud of electron-hole pairs.
For each cluster, the total energy and energy-weighted average position of the cluster are computed.
\\
The effect of the detector dead layers are computed for each step individually.
The fraction of total charge collected is modelled as a function of depth beneath the detector surface by the piecewise function
\begin{equation}
  A(x) = \begin{cases}
    0 & z < 0 \\
    g(x)=Ae^{Bx} + C & 0 \leq z < t \\
    h(x)=Mx + D & t \leq z < 1 \\
    1 & t \geq 1
  \end{cases}  \qquad
  \text{Constrained to: } \begin{cases}
    g(0)\equiv 0 \\ g(t)\equiv h(t)\equiv f \\ g'(t)\equiv h'(t) \\ h(1)\equiv 1
  \end{cases}
\end{equation}
where $x$ is the depth of the event as a fraction of the dead layer thickness, $t$ is the transition depth, $f$ is the transition fraction, and all other parameters are uniquely determined by $t$ and $f$ \cite{giovenetti2015phd}\cite{buuck2018}.
For each cluster, the uncollected charge is summed and used to compute the deadness fraction for the cluster.
The total measured energy is computed by summing the energy of each cluster, degraded by a factor of its deadness, in a single detector.
The deadness model parameters are measured by performing a fit of a simulated \Th{228} calibration spectrum to calibration data, floating the dead layer parameters for each detector.
The most sensitive parts of the calibration spectrum in this fit are the low energy portion of the spectrum, where events occur in the transition layer and have degraded energy, and in peak heights.
The parameters for this model are provided individually for each detector in a JSON file input to \texttt{process\_MJD\_as\_built\_mage\_results}.
\\
Finally, the post-processor smears energies by the response function measured during \Th{228}\ calibration runs.
The post-processor uses the peakshape functions described in appendix~\ref{appendix:peakfitter}.
Only the gaussian and low energy tail parameters are used.
The post-processor samples an energy from the probability distribution described by the peak-shape function centered at the energy calculated for the event.
The peak-shape parameters are provided individually for each detector using the input JSON file.

\subsection{Simulation Skimming}

\end{document}
