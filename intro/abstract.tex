\abstract{
The discovery of neutrino mass is the first tangible example of physics beyond the Standard Model.
Two neutrino double-beta decay (\tnbb) is an allowed second order process in the Standard Model that has been observed with half-lives in the range of $10^{18}-10^{24}$~y.
Because it involves two neutrino vertices, double-beta decay is a useful tool for studying the properties of neutrinos.
In particular, the discovery of neutrinoless double-beta decay (\znbb) would indicate that the neutrino is granted mass by the Majorana mechanism, and provide a means of measuring the mass scale of the neutrino.
This would also provide a means for violating Lepton number conservation in the Standard Model, potentially enabling a mechanism for the asymmetric creation of more matter than anti-matter in the universe.
\znbb\ has never been observed and is the active subject of a variety of experiments, with best half-life limits in the range of $10^{25}-10^{26}$~y.
In addition, parent nuclei can double-beta decay into excited states of the daughter nucleus.
Observing double-beta decay to excited states (\bbes) is helpful in understanding the nuclear matrix elements that are required for interpretting a \znbb\ result.
The branching ratios to different daughter nuclear states may also provide sensitivity to additional physics beyond the Standard Model; for example, the \tnbb\ to $2^+$ daughter states could indicate violation of the Pauli Exclusion Principle, and measurement of \znbb\ to excited states would probe the exchange mechanism underlying \znbb.

The \MJD\ is measuring double-beta decay in \Ge{76} using an array of P-type Point Contact (PPC) High Purity Germanium (HPGe) detectors.
The experiment contains 35~detectors totalling 29.8~kg that are enriched to 88\% in \Ge{76} so that the detectors act as both source and detector for $\beta\beta$-decay; there are an additional 23~detectors totalling 14.4~kg with the natural isotopic abundance.
The experiment is constructed using ultra-low background materials, in a clean environment located 4850' underground at the Sanford Underground Research Facility in Lead, SD. 
Thanks to the granularity of the detector array and the PPC dector geometry, the \Demo\ is capable of distinguishing single- and multi-site events.
The PPC detectors also have the best energy resolution of any current generation experiment, at 2.5~keV in the 2039~keV region of interist for \znbb.
These properties have enabled the experiment to measure one of the lowest background rates of currently running experiments.
The experiment is also engaged in searching for \bbes.
Excited state events are inherently multi-site due to the prompt emission of a $\gamma$-ray; by searching through events that hit multiple detectors, the \Demo\ is capable of performing a sensitive, low background search for these events.

This dissertation will begin by presenting the theoretical motivations for both the searches for \znbb\ and $\beta\beta$-decay to excited states.
Next, it will describe the \MJD, with a focus on the elements of the experiment most relevant to the search for \bbes.
The \Demo 's simulation framework will be described, along with the simulations necessary to the search for \bbes.
The techniques used to perform an optimal search for \bbes, and the estimation of the detection efficiency and its uncertainty will then be described.
A world-leading result using 22~kg-y of exposure will be presented.
Finally, this result will be placed in the context of previous results and current theory, and opportunities to improve on the result will be discussed.
}
